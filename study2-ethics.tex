\chapter{Study 2 Ethics Application}

\section{Protocol}
\subsection{Background}
The production of speech-based radio programmes often involves a significant
amount of editing, particularly for documentaries and drama. The tools used to
edit audio content are often general-purpose and not designed with speech in
mind. However, speech editing has specific demands which are not met by these
tools.

An earlier study into radio production techniques \cite{Baume2015} found that
producers go out of their way to work with speech content using textual
representations, by transcribing the recordings and working with paper
print-outs. This process adds significant overhead and cost to the production
but is considered worthwhile.

Automatic speech recognition technology makes it possible to generate timed
transcriptions of recordings, which links a text representation of the speech
to the original recording. This makes it possible to edit the audio by
manipulating a text-based interface. Evaluations of this approach
\cite{Whittaker2004, Rubin2013} have shown that it is faster and easier than
using traditional audio editing software. However, these studies have been
based in the laboratory and as such, do not take into account real-world
factors and requirements.

As part of this project, a prototype speech editing system was developed. It
provides automatic transcription and allows users to navigate and edit speech
content using a text-based interface. It also includes novel features such as
displaying where different people are speaking and the ability to export into
traditional audio editing software.

This study aims to test this prototype in a professional radio production
environment and measure its effect on the production workflow. The principal
investigator is an employee of the BBC and therefore has access to people and
resources that researchers normally would not. To make full use of this
position, participants will be recruited from BBC Radio.

The design of the experiment is based on common system evaluation and usability
study methods \cite{Lindgaard1994}. At this early stage of development, the
metrics used will be primarily qualitative. This will provide the required
flexibility to work in an uncontrolled environment and to capture reactions,
usage and demands which are unanticipated or unintended.

\subsection{Objectives}
The overall aim of this study is to measure the effect of a new text-based
audio editing workflow on the production of speech radio programmes. The new
workflow integrates the prototype speech editing system described in the
background section. The study can be divided into the following specific
objectives:

\begin{itemize}
  \item to document the existing production workflow
  \item to gather feedback on the features and usability of the prototype interface
  \item to compare the performance of the new workflow to the existing one
  \item to measure the adoption of the new workflow for day-to-day production
  \item to produce a set of design implications for informing future speech editing interfaces
\end{itemize}

\subsection{Hypotheses}
This study is designed to be exploratory. As such, there may not be enough
participants to fully test the following hypotheses. However, they are included
here as aspirations in the longer-term.

Compared to the existing method, text-based audio editing of speech for radio
programmes:
\begin{itemize}
\item is faster
\item has less cognitive load
\item is preferred by users
\end{itemize}

\subsection{Participants}
\subsubsection{Recruitment}
To make the most of the principal investigator's position in the BBC,
participants will be recruited exclusively from BBC Radio. They will be invited
to volunteer by way of email passed around using existing contacts in various
departments. 

The nature of the study requires that participants conduct the tasks as part of
their day-to-day job. Therefore, participants will be required to gain
permission from their line manager (known in radio as 'editors') to take part.

Programmes and producers vary significantly in genre and production techniques.
To take this into account, participants will be recruited so that there is a
representative range of styles.

\subsubsection{Selection criteria}
\begin{itemize}
\item The radio programme being created should be speech-based.
\item The content of the programme must not contain any sensitive material, as
the recordings will be sent to a third-party for transcription (see 'data
protection' section below).
\item The turn-around time of the programme must be long enough to allow time
to recover from any technical issues without affecting broadcast output.
\item The participant must have permission from their editor to take part.
\end{itemize}

\subsubsection{Number of participants}
The study will involve between five and eight participants, depending on how
many meet the criteria. This should uncover roughly 85-95% of usability
problems \cite{Nielsen1993} whilst covering a range of genres and styles, and
being a manageable number given the duration of the experiment.

\subsection{Location}
The study will take place at the participant's own desk at their normal place
of work. As the prototype is web-based, they can use their own computer. There
is a standard configuration for BBC desktop computers, so the environment
should be similar for each participant. Use of display screen equipment is
already covered under the existing BBC risk assessment process.

\subsection{Experimental design}
\subsubsection{Briefing and pre-interview (30 mins)}
The objective of this stage is to ensure that the participant is familiar with
the study and its aims, and is happy to proceed. They will be briefed on the
background, objectives and design of the study as detailed in the protocol.
Should they wish to participate, they will be asked to read and agree to the
consent form.

The participant will be asked about their production style and the programme
that will be observed. Aspects that are of interest at this stage include:
\begin{itemize}
\item Participant's production experience
\item Use of transcription
\item Use of paper
\item Turn-around time
\item Editing tools
\item Features of their existing workflow they do or don't find useful
\end{itemize}

As some radio programmes can take over three weeks to create, the experiment
will only cover a representative task within the programme's production, rather
than the programme in its entirety. The experimenter will work with the
participant to select a suitable task for the experiment, and define a clear
start and end point. An example of such a task would be editing a long
interview.

The criteria for selecting the task is:
\begin{itemize}
\item must primarily involve editing of speech
\item must take no longer than two days to complete
\item must be repeated with different but similar content at least once
\end{itemize}

Finally, the experimenter and participant will agree on a schedule of dates and
times for the next stages of the experiment.

\subsubsection{Interface training and usability study (30 mins)}
There are two objectives to the training stage – firstly to introduce the
participant to the prototype system so that they understand its capabilities,
and secondly to test the usability of the interface.

A user account for the participant will be created on the prototype system and
they will be taken through a 'tooltip' demonstration that highlights the
different features of the interface and explains their function. Once that is
complete, the usability testing will begin.

The participant will be asked to perform a series of typical tasks, listed
below. The participant can ask questions and, if they become stuck, the
experimenter can prompt them. However, the experimenter will not give any other
directions. The experimenter will note whether the participant was able to
successfully complete the task and document any problems or confusion
experienced during the execution of the tasks. This process will help to flag
any obvious stumbling blocks.

\begin{enumerate}
\item Upload two audio recordings provided on a USB key, labelling them with a given name and accent.
\item Play, pause and stop the recordings.
\item Skip to a specific time.
\item Skip to a specific phrase.
\item Select and clip specific phrases in each recording.
\item Play/pause/stop the clips.
\item Change the order of the clips.
\item Delete one of the clips.
\item Export the new edit into an audio editor.
\end{enumerate}

\subsubsection{Real-time observation (2-4 days)}
This stage will involve observing a task that was agreed at the briefing stage.
The task will be observed twice – once using the existing production method,
and again using the prototype system. The order of the observation will be
alternated between participants. Similar but different audio content will be
used for each task so that the participant isn't already familiar with the
content.

The observation will be done passively to allow natural interaction with the
system in the participant's normal work environment. The experimenter will sit
beside them making written notes. When the participant has some 'down-time',
the experimenter may choose to ask some questions in order to verify or clarify
something they have observed.

The specific items of interest at this stage include:
\begin{itemize}
\item People and their roles
\item Editing workflow
\item Tools used
\item Data generated (both digital and paper-based)
\item Usability challenges and problems
\item Audio navigation and edit actions
\item Time taken to complete tasks
\item Unexpected reactions
\item Unanticipated usage
\end{itemize}

The prototype will be configured to electronically log the following actions,
which will help provide insights into usage of the various features of the
prototype:
\begin{itemize}
\item Audio file upload (incl. duration)
\item Transcription upload
\item Play, pause and stop
\item Navigation-by-word
\item Navigation-by-time
\item Drag-and-drop (edit)
\item Re-order edits
\item Delete single edit
\item Delete all edits
\item Export edit (incl. format)
\item Create project
\item Delete project
\end{itemize}

Immediately after completing the task, the participant will be asked to rate
their experience using the NASA Task Load Index \cite{Hart1988} metrics. This
will allow a comparison of the cognitive load of each experimental condition.

With the participant's permission, photos may be taken of paper notes,
annotations or the work environment (see 'data protection' section below).

\subsubsection{Post-interview (1 hour)}
After both observations are complete, the participant will be interviewed about
their experience. The primary objective is to directly compare the two
production methods and extract the advantages and disadvantages of both.

An audio recording will be made of the interview (see 'data protection' section
below). This will allow the participant to speak their mind and allow the
experimenter to give the participant their full attention whilst capturing all
of the provided information.

The questions asked will include:
\begin{itemize}
\item Which aspects of the existing system did you / did you not find useful?
\item Which aspects of the prototype system did you / did you not find useful?
\item Overall, which system did you prefer and why?
\item Did the prototype change the way you made the programme? If so, how?
\end{itemize}

\subsubsection{Follow-up (2-month period)}
Once the experiment is complete, the participant's user account on the
prototype system will remain active and they will be allowed to continue to use
it on their own, should they wish. Each week for two months, they will be
emailed a short questionnaire intended to collect information about their
continued usage and experience of the system, or lack thereof.

The questions will include:
\begin{itemize}
\item Have you used the prototype system in the past week?
\item If you did use the prototype system:
\item How many hours did you use it for?
\item What did you use it for?
\item Which features do you value the most and why?
\item Which features do you value the least and why?
\item Are there any missing features you would like to see implemented?
\item If you didn't use the prototype system, why not?
\end{itemize}

The electronic data logging described in the real-time observation section will
continue to run during this stage, collecting information about system usage
which should provide further insight into the frequency of usage and popularity
of features.

At the end of the two months, the participant's account will be disabled.

\subsection{Data protection}
\subsubsection{Anonymity and personal information}
The names of participants will not be published, nor will the names of the
programme they are working on. However, the genre or topic of the programme
will be used to provide an appropriate context.

Each participant that takes part in the study will be assigned an ID number at
the time they sign their consent form. All information and material collected
during the experiment will be linked to the participant's ID number. The
consent form, which links their identity to their ID, will be stored by the
principal investigator in a secure location at the BBC.

\subsubsection{Production content}
All audio files that are uploaded to the prototype system will be transcribed
by a third-party (Cantab Research Limited trading as Speechmatics). The privacy
policy for the transcription service can be found at
https://www.speechmatics.com/privacy. The audio will be uploaded to their
servers, transcribed and then immediately deleted. Every time a participant
uploads an audio file, they will be required to read and agree that they
understand the following statement:

``Your files will be sent to a third party for automatic transcription. Every
effort will be made to protect your data, however we cannot guarantee that this
service is secure. Do not upload any material that should not be in the public
domain.''

Audio files which are uploaded to the prototype system will be stored on
BBC-managed servers which can only be administered from within the premises of
BBC Research and Development. Administrative accounts will be limited to the
principal investigator and the IT staff that manage the servers.

Access to uploaded files and their derivative content (e.g. transcriptions)
will be limited to the administrators and the participant themselves. Access to
the system is authenticated using the BBC iD system (see
https://ssl.bbc.co.uk/id/info).

\subsubsection{Written notes}
The notes taken by the experimenter during the briefing, training and
observation stages may be made publically available, but the identity of the
participants will be anonymised.

\subsubsection{Questionnaire}
The anonymised raw responses to the follow-up questionnaire may be made
available internally within the BBC, but will only be published publically in a
summarised form.

\subsubsection{Photographs}
Photos of production material and the work environment may be taken, but only
with the participant's explicit permission. To protect the participants and
their colleagues, the photos will not contain any recognisable individuals. The
photos will be stored by the principal investigator in an encrypted format.
They may later be used as figures in publications, but only with explicit
consent from the participant.

\subsubsection{Audio recording}
An audio recording of the post-interview will be made. The data will be stored
in an encrypted form by the principal investigator. The audio and transcript
from the recordings will not be published, but may be made available internally
within the BBC. Information gathered during the interview may be published
anonymously in a summarised from, such as a short quote for an academic paper.

\subsubsection{Data logging}
The prototype system will store a log of the interactions of each participant
(see 'real-time observation' section). This will be securely stored on the same
BBC-managed servers as the production content. The raw logs may be made
available internally within the BBC, but will not be made available publically.
However, a summary of the data may be published.

\subsection{Evaluation and analysis}
\subsubsection{Usability study}
The notes taken by the experimenter during the usability study will be analysed
to identify common usability issues experienced by participants whilst
executing the assigned tasks. Any obvious themes will be identified and a set
of suggested design changes will be produced.

\subsubsection{Pre-interview and real-time observation}
The notes from the pre-interview and real-time observation stages will be
analysed to:
\begin{itemize}
\item document the existing workflow that is used to create radio programmes.
\item identify the new workflow that emerges from using the prototype system.
\item compare the performance of the existing and new workflows.
\item highlight any unexpected or unintended reactions or usage.
\end{itemize}

\subsubsection{Post-interview}
The audio recordings from the post-interview will be transcribed and subjected
to thematic analysis \cite{Braun2006}. The responses will be segmented into
different topics which are assigned a code. The occurrence of each code and
their interactions will be then analysed. The responses will also be studied to
identify any unexpected responses or unanticipated usages which may inform
further work.

\subsubsection{Follow-up}
The responses from the follow-up questionnaires will be used to track usage of
the prototype over time. It will also be used to identify which features are
the most and least important, and to discover any longer-term usability issues.

\subsubsection{Data log}
The interaction logs collected during the observation and follow-up stages will
be summarized using the following metrics:
\begin{itemize}
\item System usage patterns over time (e.g. number of new projects)
\item Feature usage (e.g. number of exports)
\item Relative usage of similar features (e.g. navigation-by-word vs. navigation-by-time)
\end{itemize}

Graphs displaying the distribution of the metric results will be produced to
see if there are any obvious trends. However due to the qualitative nature of
the study and the small number of participants involved, it is not anticipated
that statistically significant results will be found at this stage.

\section{Consent form}
\begin{itemize}
\item I the undersigned voluntarily agree to take part in this study on
text-based editing of speech radio.
\item I confirm that I have received permission from my line manager to take
part in this study.
\item I agree to comply with any instruction given to me during the study and
to co-operate fully with the investigators.
\item I have read and understood the Information Sheet provided. I have been
given a full explanation by the investigators of the nature, purpose, location
and likely duration of the study, and of what I will be expected to do. I have
been given the opportunity to ask questions on all aspects of the study and
have understood the advice and information given as a result.
\item I consent to my personal data, as outlined in the accompanying
information sheet, being used for this study and other research.  I understand
that all personal data relating to volunteers is held and processed in the
strictest confidence, and in accordance with the Data Protection Act (1998).
\item I understand that any audio I upload to the prototype system will be sent
to a third party to be transcribed, and that there is a small risk it could be
made public. 
\item I consent to the audio from the post-interview being recorded, and
understand that photographs may be taken and used with my permission. 
\item I understand that I am free to withdraw from the study at any time
without needing to justify my decision and without prejudice.
\item I confirm that I have read and understood the above and freely consent to
participating in this study.  I have been given adequate time to consider my
participation and agree to comply with the instructions and restrictions of the
study.
\end{itemize}

\section{Information Sheet}
\subsection{Introduction}
My name is Chris Baume and I would like to invite you to take part in a study
for my PhD thesis. This information sheet will explain what I’m trying to
achieve and how you can get involved. Before you decide whether to take part,
please take the time to read the following information carefully. Feel free to
ask me any questions and talk to others if you’re unsure of anything.

\subsection{What is the purpose of the study?}
This study aims to investigate how a text-based editing workflow affects the
production of speech-based radio programmes.

\subsection{Why have I been invited to take part in the study?}
You are the producer of a radio programme which involves a significant amount
of editing speech recordings.

\subsection{Do I have to take part?}
No, you do not have to participate. There will be no adverse consequences to
your employment if you decide not to participate. You can withdraw at any time
without giving a reason.

If you choose to withdraw, any data collected during the trial that could link
back to you will be destroyed (i.e. audio recordings, photographs), but
anonymous data may be retained and used for analysis (i.e. examiner’s notes,
interaction logs, questionnaires).

\subsection{What will my involvement require?}
To take part, you need permission from your line manager.

There are five stages to the study, outlined below:
\begin{enumerate}
\item Briefing and pre-interview (30 mins)\\
I will explain what the study is about and what it involves before asking you
some preliminary questions about your experience, production style and the
tools you use. We will work together to identify a typical task which will be
observed at a later stage.
\item Training and usability study (30 mins)\\
You will be introduced to the prototype system and asked to use it to complete
a series of simple tasks. I will sit beside you whilst you use it and make
written notes on whether you run into any issues with the interface.
\item Real-time observation (2-4 days)\\
I will sit with you at your desk while you complete the task identified in
stage one as part of your normal job. The task will be observed twice – once
using your existing workflow and the other using the new text-based workflow. I
will take written notes throughout based on what I observe, and your activity
on the prototype system will be logged electronically. I may ask to take
photographs on occasion, but this will only be done with your consent.
\item Post-interview (1 hour)\\
After you have completed both tasks, I will ask you a series of open questions
about your experience of the two different workflows, and ask you to compare
them directly. The audio from the interview will be recorded so that I can
later recall everything that was said.
\item Follow-up (2-month period)\\
You will be given a user account on the prototype and left alone. You are
welcome to continue to use the prototype, but are under no obligation to do so.
I will email you once a week with a short questionnaire on your usage of the
prototype. Your activities on the prototype will continue to be logged
electronically.
\end{enumerate}

You will not receive any payment or compensation for taking part.

\subsection{What are the possible disadvantages or risks of taking part?}
The prototype system uses a third-party service to automatically transcribe the
recordings you upload to it. This means that any uploaded audio is copied to a
server outside of the BBC (and the UK) where it is transcribed then deleted.
Although every effort will be made to protect your data, it is possible that
the recordings or their transcription may be intercepted. For this reason, we
ask that you do not upload any sensitive material that would cause problems if
made public.

Other than the transcription functionality, the system is hosted by the BBC and
all possible precautions are being made to keep your information secure. It is
not anticipated that you will experience any disadvantages by taking part in
this study.

\subsection{What are the possible benefits of taking part?}
By taking part in the study, you will be contributing to valuable research on
radio production tools. It is hoped that this will result in improved workflows
for radio producers in the near future.

\subsection{What will happen to the information I provide?}
All the data you provide will be anonymised so that those reading reports from
the research will not know who has contributed to it. This is done by removing
your name and the name of the programme, however the genre or topic of the
programme may be included to provide context.

The notes I take during observations and interviews may be made publically
available. Answers to the questionnaire, the audio and transcripts from the
post-interview and the system usage data may be made available internally
within the BBC, but will be only be made public in a summarised form. Any
photos taken will only be published with your permission. The prototype system
is hosted within the BBC, and access to your data is restricted to you, me and
the system administrators.

Personal data will be handled in accordance with the Data Protection Act 1998.
All data will be stored for a minimum of 6 years, and any data considered
relevant to the findings of the study will be stored for at least 10 years. The
data will be stored securely at the University of Surrey.

\subsection{What happens when the research study stops?}
At the end of the study, your responses will be analysed alongside the other
participants’ responses to look for common themes or anything unexpected. The
results will be submitted for publication at academic conferences or journals,
and may be disseminated through other media such as blogs. You will be sent
links to any publications that result from the study.

After the follow-up period is complete, your access to the prototype may be
revoked so that the system can be improved.
 
\subsection{How do I get in touch or make a complaint?}
Please direct any questions, concerns or complaints to me using the following
details:

Email: chris.baume@bbc.co.uk
Telephone: 03030 409683
Post: BBC R\&D, Centre House, 56 Wood Lane, London, W12 7SB

If you need to speak directly to my supervisor, Prof. Mark Plumbley, please use
the following details:

Email: m.plumbley@surrey.ac.uk
Telephone: 01483 689843
Post: Centre for Vision Speech and Signal Processing, Faculty of Engineering and Physical Sciences, University of Surrey, Guildford, Surrey, GU2 7XH

Any complaint or concern about any aspect of the way you have been dealt with
during the course of the study will be addressed.

\subsection{Who is organising and funding the research?}
This research is fully funded by the British Broadcasting Corporation.

\subsection{Who has reviewed the project?}
The study has been reviewed and received a Favourable Ethical Opinion (FEO)
from the University of Surrey Ethics Committee.

\section{Invitation letter}
Dear xxx,

My name is Chris Baume. I’m a research engineer at BBC R\&D and a PhD student at
the University of Surrey, working on developing tools for radio production. As
part of my work, I have created a prototype system which automatically
transcribes recordings and allows you to edit using text instead of waveforms.
I am looking for volunteers to test out this new system in the production of a
radio programme and to provide feedback.

To take part, you must have the permission of your line manager and be working
on a suitable programme. The programme schedule must be relaxed enough that any
technical issues can be rectified and the content should not be of a sensitive
nature, as the prototype must send the audio to a third-party for
transcription. The study will involve three short interviews, a few days of
observation while you produce the programme and a series of follow-up
questionnaires. 

If you are interested and think you may be eligible, please let me know and I
can send you more information about what is involved and answer any questions
you may have.

Many thanks,

Chris Baume

\section{Risk assessment statement}
This study will take place at BBC Broadcasting House in London, where the
participants will be completing the study as part of their job. They will be
physically located either at their normal desk or in a meeting room within the
building. The experiment will only be introducing new software, so the
participants will be using their existing hardware in the same manner as they
did previously.

As this is a live working environment, the risks associated with operating in
this building are already covered under existing risk assessments which are
regularly updated and maintained. These assessments cover fire, first aid and
occupational health amongst other things. Any mitigations identified by the
identified risks will have already been implemented.

For most of the study, the participant will be using a desktop computer at
their normal desk. Each participant will have already completed a ‘display
screen equipment’ (DSE) assessment which checks the suitability of their desk
configuration, chair, equipment and positioning. This is a compulsory exercise
for using a computer within the BBC, and the assessment must be repeated
annually.

As the study is bring conducted in an environment which has already undergone a
thorough risk assessment, and does not introduce any new equipment, it is the
position of the investigators that the study does not need a separate risk
assessment.
