% !TeX root = main.tex
\chapter*{Abstract}

Radio production is a creative pursuit that uses sound to inform, educate and entertain an audience. Radio producers
use audio editing tools to visually select, re-arrange and assemble sound recordings into programmes. However, current
tools represent audio using waveform visualizations that display limited information about the sound.

Semantic audio analysis can be used to extract useful information from audio recordings, including when people are
speaking and what they are saying. This thesis shows how such information can be applied to create semantic audio tools
that improve the radio production process.

An initial ethnographic study of radio production at the BBC reveals that producers use textual representations and
paper transcripts to interact with their audio content, and waveforms to edit their programmes. Based on these
findings, three methods for improving radio production are developed and evaluated, which form the primary contribution
of this thesis.

Audio visualizations can be enhanced by mapping semantic audio features to colour, but this approach had not been
formally tested. We show that with an enhanced audio waveform, a typical radio production task can be completed faster,
with less effort and with greater accuracy than a normal waveform.

Speech recordings can be represented and edited using transcripts, but this approach had not been formally evaluated
for audio production. By developing and testing a semantic speech editor, we show that automatically-generated
transcripts can be used to semantically edit speech in a professional radio production context, and identify
requirements for annotation, collaboration, portability and listening.

Finally, we present a novel approach for editing audio on paper that combines semantic speech editing with a digital
pen interface. We compare the relative benefits of using paper and screen interfaces for semantic speech editing, and
gain insights into the effect of transcription errors and the reasons for listening.

