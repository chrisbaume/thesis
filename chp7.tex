% !TeX root = main.tex
\chapter{Conclusions and further work}\label{chp:conclusions}

The aim of this research was to ``develop and evaluate methods for radio production that improve the production
process'' (Section~\ref{sec:aim}).  We focused our research on pre-production of speech content by professional radio
producers to make the most of the access available to us from working within the BBC.  In fulfilment of our aim, the
primary contribution of this thesis has been the development and evaluation of three methods for editing speech
recordings, through audio visualization, semantic speech editing and a digital pen interface.  We developed these
methods based on genuine requirements gathered from radio producers and evaluated them in the workplace to
ensure that our methods and results were relevant to real-life application.

To conclude this thesis, we first discuss our approach, results and contributions in
Section~\ref{sec:conclusions-discussion}, where we also reflect upon some of the tensions we observed between reading
transcripts and listening, and between paper and screen interfaces.  In Section~\ref{sec:conclusions-further}, we
describe potential options for further work, including follow-up research resulting from our studies, as well as some
broader applications of semantic audio production tools.  Finally, in Section~\ref{sec:conclusions-conclusions} we
summarise the novelties and achievements of this work.

%\section{Radio production practice}
\section{Discussion}\label{sec:conclusions-discussion}

We began our research by conducting three ethnographic case studies in the workplace to learn more about real-life
radio production practice.  We used the results to develop theoretical models of production workflows for a news
bulletin, drama and documentary.  We developed these based on direct observation of actual practice, which gave us 
insights into the genuine processes and challenges of radio production.  In addition to the workplace studies in
Chapter~\ref{chp:ethno}, we deepened our understanding of existing production workflows through interviews with twelve
radio producers as part of the user studies in Chapters~\ref{chp:screen} and \ref{chp:paper}.  These models and
insights contribute to the academic understanding of radio production practice.  The results of this ethnographic work
highlighted three directions for research involving audio visualization, textual representation of speech, and the use
of paper.  We then investigated each of these through technical intervention.

\subsection{Semantic audio visualization}

% Coloured waveform was better
Our initial investigation looked at using pseudocolour to visualize a semantic audio feature to support audio editing.
We measured the user performance for an editing task using our semantic audio visualization, compared to a normal
waveform.  The results showed that when using the semantic audio waveform, the participants completed the task faster,
with less effort and with greater accuracy than the normal waveform. This demonstrates that there is value in the
pseudocolour approach to semantic audio visualization taken by \citet{Rice2005}, \citet{Akkermans2011} and
\citet{Loviscach2011a}, which had previously been untested.

% Limited to one task, many more applications
Our experiment only focused on a single task of segmenting music from speech so there are many opportunities for
applications beyond the task we chose. Audio visualizations can either target specific tasks, or
general use that covers a range of tasks. Focusing on an individual task may produce better results, but are only suitable
for that task.  Designing a general audio visualization is much more challenging, but has the potential to create a 
greater impact as it could be used for a variety of applications.

% Chose rudimentary feature, could do better
The semantic audio visualization we designed and tested used a rudimentary semantic audio feature rather than the
state-of-the-art.  We did not attempt to create the best possible visualization as we wanted there to be an element of
human judgement in the measured task. There is potential to make better visualizations by including more and
better semantic audio features, and using more advanced methods of visualization. For example, the false colour
approach taken by \citet{Tzanetakis2000} and \citet{Mason2007} allowed multiple features to be displayed simultaneously
in a human-readable way.

% Normal waveforms not great
Although we found that users required less effort to complete the task with a normal waveform than without, they did
not complete the task significantly faster nor more accurately. We found this surprising as waveforms are widely used
throughout almost all audio editing software, so it was expected that waveforms would improve the performance of users in
completing audio editing tasks. The performance of waveforms significantly impacts the audio production community, so
this poor performance is concerning. However, this finding highlights an opportunity to increase the efficiency of
audio production software by making improvements to the waveform visualization.

\subsection{Semantic speech editing}

% can be used for pro production
We conducted two user studies in which semantic speech editing was successfully used by professional radio producers to
create real radio programmes that were subsequently broadcast. Our results support previous work in finding that
semantic speech interfaces help users navigate and edit speech \citep{Whittaker2002}, and that semantic editing is
faster than, and preferable to, using audio waveforms \citep{Whittaker2004,Sivaraman2016}. Our research went further by
conducting user evaluations in a natural working environment, and directly comparing semantic speech editing to the
existing editing workflow. This allowed us to gain insights into its limitations in the context of radio production.

The efficiency savings realised from the use of semantic speech editing tools and ASR could provide cost and time
savings over traditional editing and manual transcription.  This should free up time and budget for more valuable
production activities, which may lead to improvements in the quality of programmes and/or reduction in the cost of
production.

% ASR vs verbatim
Our semantic speech editor was similar to the system described by \citet{Rubin2013}. However, rather than using
verbatim transcripts, which are slow and expensive to produce, we used automatic speech recognition (ASR). ASR is
better suited for use in broadcast, but the erroneous transcripts it produces affect the usability of semantic editing.
Our studies revealed that the accuracy of the transcript affected the need for correction, reading speed, reliance on
listening, longevity of the transcript and the edit granularity.  Crucially, we discovered that the quality of 
modern ASR systems was sufficient to allow for semantic speech editing in radio production.  This aligns with similar
findings for the semantic editing of voice messages \citep{Whittaker2004,Sivaraman2016}.

% Only good for rough edits, but integrated with DAW
The producers we tested reported that they only found semantic editing useful for creating a ``rough edit'', where
large segments of audio are selected from the original material. This contradicts findings by \citet{Sivaraman2016} for
the editing of voice messages.
%found that for lightweight
%voice editing, users were mainly interested in using semantic editing to make fine-grain edits.  Our research showed
%that in the context of radio production, semantic editing tools were used differently.
We found that the main reason for this limitation was the lack of annotation and re-ordering features, which made it
difficult for producers to organise, structure and arrange their material in the later stages.  However, this
limitation was not an issue as our semantic speech editing tools integrated with several digital audio workstations
(DAWs), which allowed the producers to seamlessly transition to their normal tools to complete their production.

%seamlessly move from our tools to their normal editing
%software, so that they could use semantic editing tools to create a rough edit, then

%by exporting an edit decision list
%(EDL) that described the producer's edits. This allowed for a non-destructive integration where edits could be undone.

% correction
We tested a variety of additional features to support semantic editing, including transcript correction and confidence
shading.
%TODO and speaker diarization.
We found that many of the transcript errors encountered were specific to the
programme content, such as names and topic-specific words.  However, most producers were only interested in correcting
errors that were particularly distracting.  Most producers we tested found confidence shading to be useful overall,
which supports \citet{Burke2006}, but contradicts \citet{Suhm2001} and \citet{Vemuri2004}.

%Both semantic speech editors included confidence shading, which low-lighted or underlined words that the ASR was not
%confident were correct.


% APPLICATIONS


%Radio producers often work to tight unmovable deadlines, and have a small and shrinking budget.
%This extra time could be
%spent on more valuable activities such as research and recruitment of interviewees, which may improve the overall
%quality of the final programme. Alternatively, these efficiencies could be used to reduce the cost of production whilst
%retaining the same quality of output.

DAWs are powerful, feature-rich tools, but to novice users they can be intimidating to work with.  Semantic speech
editing tools are easier to use and more accessible than waveform-based tools \citep{Yoon2014,Sivaraman2016}.
Podcasting has already democratised the distribution of audio content, and semantic speech editing may have the
potential to achieve something similar for audio production. This would allow more people to access audio production,
which would hopefully lead to the production of more and better audio content.

Alternative methods of interacting with audio content may also give rise to new creative opportunities for more
innovative programme making.  For example, the comedy duo known as ``Cassetteboy'' \citep{Perraudin2014} patiently
trawl through hours of video footage to re-order the words of famous speeches in amusing ways, which could be achieved
much more easily using a semantic speech editor.

%Semantic speech editing provides an alternative to audio waveforms for interacting with speech content. Text-based
%interfaces allow users to quickly search and discover content. This may open up new possibilities for different types
%of content that were not previously possible when using waveforms. For example, the comedy duo ``Cassetteboy'' remix
%speech content of popular figures to change the content in radical ways, such as turning political speeches into rap
%songs. Doing this manually requires huge amounts of patience and time to edit, but could be achieved relatively quickly
%using a semantic speech editor.

\subsection{Reading vs listening}

% Limitations of transcripts
Transcripts display the words that were spoken in a recording, but much of the content's true meaning is hidden in the
audio. Although transcripts show \textit{what} was said, they do not reveal \textit{how} it was said, which is crucial
when producing radio programmes.  As one producer pointed out, ``radio is made with your ears''.

% reasons for listening
The only way to fully comprehend audio recordings is through listening.  We found that radio producers used listening
to process information, judge the quality of the sound and identify non-speech sounds.  Some producers reported that
reading and listening at the same time enabled them to comprehend information more easily, supporting the findings from
\citet{Vemuri2004}.  Producers listened to identify any low quality audio, such as ``umm''s and ``err''s, and long or heavy
breaths. These are visible in verbatim transcripts, as used by \citet{Berthouzoz2012} and \citet{Rubin2013}, but are
not included in ASR transcripts.  Producers also listened to identify any high quality moments that may not be
identifiable using the transcript, and non-speech sounds that they might want to include or exclude in their programme.


% not listening while logging
The existing radio production workflow involves producers ``logging'' recordings by listening to the audio and typing
rough quotes and labels.  This process is time-consuming and tedious, but it helps the producer to review and organise
their content.  ASR transcription replaces this logging process, which saves producers time and effort. However, it
also means that producers don't have to listen to the audio before editing, which may prevent some producers from
reviewing the material before making editorial decisions.  There is a risk that over-reliance on transcripts may mean
missing out on good quality content or not avoiding poor quality content, which would affect the overall programme
quality.

%Some of the producers in the Chapter 5 study
%said they valued this listening opportunity, and one of the producers in Chapter 6 wanted to listen back in full first.

% integrated playback
Providing an easy way to listen to the audio underlying the transcript is important. Our screen-based semantic speech
editor included integrated playback, which allowed users to navigate the audio using the text.  Time compression, such
as described in Section 2.6, would also allow producers to listen faster, and producers valued this feature when we
added it to our screen-based editor following user feedback. Reading the transcript while listening also increases the
speed at which time-compressed audio can be comprehended \citep{Vemuri2004}.

% transcripts not always needed
However, we found that in some situations, transcripts are not necessary or useful. With short recordings of less than
15-25 mins, producers reported that they can remember what was said and when, and some programmes have a greater focus
on the sound design, which limits the value of a transcript.

%However, our paper-based
%editor did not, which meant that users did not cope well with low accuracy transcript.

% use of time compression

We found that transcripts are not normally published due to the high cost of creating verbatim transcripts.  ASR and
transcript correction tools could make it possible for producers to create a verbatim transcript as part of their
production process.  Publishing these could allow audio content to be more easily searchable and discoverable through
Internet search engines, for example.  Word-level timings could also be used to link directly to segments of audio. For
example, NPR's online clipping tool ``Shortcut'' \citep{Friedhoff2016} allows radio listeners to create a short clip
from a radio programme and share it with their friends. 

\subsection{Paper vs screen}

Our workplace studies identified that many radio producers enjoy working with paper.  Our user study in
Chapter~\ref{chp:screen} also found that many producers want to be able to work away from their screens.  This led us
to develop a novel semantic speech editing system that used digital pens to allow producers to edit speech recordings
using a paper transcript.  We conducted a qualitative user study of professional radio producers that compared our
paper interface to a screen-based semantic speech editor. The results of our study provide insights into the relative
benefits of editing speech on paper compared to screens. We found that both the paper and screen interfaces worked
well, but that each had advantages and disadvantages in different contexts.

% summary
Overall, we found that the paper interface was better suited to quick and simple edits where listening is not as
critical, such as with high accuracy transcripts, or very recent recordings.  Our screen-based semantic speech editor
included integrated playback and correction features, which made it better for more complex editing with less familiar
audio and less accurate transcripts.

% benefits of paper
Producers reported that paper transcripts were easier to read and remember, and made it easier for them to think widely
and orientate themselves.  They reported that our digital pen interface was simple, intuitive, precise and allowed edit
decisions to be made quickly and easily.  However, producers had concerns over the cost of an additional device which
would likely have to be shared amongst a team, and because they were worried about losing it.
%TODO Why is sharing a problem?

% portability
The physical nature of the digital pen and paper made it better suited to travel and working away from the desk. This
gives producers greater flexibility to work in more comfortable locations and while commuting.  However the digital pen
interface uses considerable amounts of paper, involves carrying a digital pen, and requires access to a colour laser
printer, which would make it difficult to work ``on the road''.  The screen interface could be used on a
laptop or tablet, which have the added benefit of integrated playback, but screens are heavier, bulkier and have a much
shorter battery life than digital pens.

% collaboration
Radio is usually produced by a team, so tools that facilitate collaboration are important to working in such an
environment.  We found that the physicality and pagination of paper made it better suited to face-to-face collaboration
than the screen. However, the screen interface is capable of remote collaboration, and could be used over the Internet
for real-time collaborative speech editing.

% implementation limitations - integrated playback, correction, undo
The limitations of the digital pen system we used prevented us from including features for integrated playback,
correction and undo. The lack of integrated playback forced producers to replay and navigate audio using a separate
device, which made it more challenging to interpret errors in the transcript. This was more of an issue with low
quality transcripts, which could also not be corrected using the digital pen.  The integrated playback of the screen
interface made it easier for the participants to find and fix mistakes in the transcript.  This also made it
easier to edit content that required more listening, such as old or unfamiliar recordings.

%Our system interpreted the user's gestures strictly, which created a source for unintentional errors.

%\section{Applications}


%The recent release and investment in the Descript editor demonstrates that there is increasing commercial interest in
%semantic speech editing for audio production. Avid have included their ScriptSync tool since 2007(?) and released a
%major update in 2016(?). This shows that there is market demand for timed transcript tools and semantic speech editing
%in the audio production community.



%=====================================================================================================================

% CHAPTER INTRO


% Release the structure of the chapters

%=====================================================================================================================

% RE-INTRODUCTION
%   - an introductory restatement of research problem, aims and/or research question
%   - restate the objectives


% APPROACH - how and why - justify the approach adopted

%To achieve this aim, we started by conducting three ethnographic case studies to investigate how radio production is
%currently conducted at the BBC. This helped us to ensure that the direction of the research was aligned with the
%real-life challenges faced by professional producers.  This investigation provided three directions for the research,
%each of which we investigated in turn.

%Previous work had used audio visualization techniques to display more information about audio signals, but they had not
%attempted to measure the effect these visualizations had on production tasks.
%We conducted a quantitative study that measured the actual and reported performance of users in segmenting music from
%speech using different audio visualizations.
%This allowed us to demonstrate the measurable effect that waveforms and a simple semantic audio visualization had on
%user performance of a typical radio production task.

%Semantic speech interfaces had been considered by previous research as an alternative method of navigating and
%editing speech content.
%These systems did not attempt to measure user performance, and most used perfect transcripts which are too expensive to
%produce in professional production. \citet{Whittaker2004} showed that ASR transcripts could be used for editing.
%We developed a semantic speech editor that integrated with the BBC's radio production systems, which allowed us to
%evaluate semantic editing in a professional context.
%We then evaluated our system on radio producers who used it to produce programmes for broadcast, and compared our
%system to the existing workflow.

%Finally, we identified an opportunity to develop a novel interface that used paper transcripts to edit speech content.
%We worked directly with professional producers to refine an early prototype, then collaborated with a digital pen
%manufacturer to create a working system.
%We evaluated our system for use in real-life programme-making, and compared it to our screen-based semantic speech
%editor, and an inactive paper transcript.

% TAKE-HOME MESSAGE - highlight the ‘take-home message’ 

%Professional radio production can benefit from semantic audio tools

%We demonstrated three approaches in which semantic audio can be used to make professional radio production more
%efficient

%Audio visualization and semantic speech editing have benefits to professional radio production




% SUMMARY
%   - a summary of findings and limitations
%   - Summarize what was said in the different chapters
%   - include the most important points already covered

%Chapter~\ref{chp:ethno} used ethnographic case studies to investigate audio editing workflows for the production of a
%news bulletin, drama and documentary within BBC Radio. Our investigation identified three themes regarding audio
%waveforms, textual representaton and the use of paper.

%- Audio waveforms are widely used
%- Producers work with textual representation
%- Drama and docs producers use paper

%Chapter~\ref{chp:colourised}:

% Audio visualization
% - previous systems had not been formally evaluated
% - Using semantic audio to add colour to waveform improves performance for music/speech segmentation
% - Faster, more accurately and with less effort
% - our study demonstrates pseudocolour approach by Rice2005, Akkermans2011 and Loviscach2011a has value
% - false colour from Tzanetakis2000 and Mason2007 may have greater benefits

%- Normal waveform was not significantly faster, nor more accurate
%- We chose task-specific, but alternative could be general
%- We only focussed on one task. Audio visualization has applications beyond music/speech segmentation, such as
%  programme segmentation, spotting noises


%Chapter~\ref{chp:screen}:

%- Semantic speech editing can be used for professional radio production (continued to use)
% - Whittaker2004 found that semantic editing of voicemail messages was faster than editing with a waveform. Our
%   research supports this, and shows that it also applies to professional radio production
% - Sivaraman2016 found semantic editing was good enough to replace waveform editing and more accessible

%- The goal of our system was similar to Rubin2013, but used ASR, integrated with radio production system and was evaluated
%- Annotation is needed to organise and structure content
%- Collaboration is useful
%- Portability allows flexible working away from desk
%- Does working with text affect the final result? Could have negative impact, could open up creative possibilities

% radio is made with your ears
% - text can only represent the words that were spoken, which only tells half the story
% - semantic editing can't fully repace it
% - combination with listening features such as time compression could help

%- Listening is important
% - limits usefulness of semantic speech editing
% - Whittaker2002 found that users could use stratigic fixation to scan transcripts to extract info and judge
%   relevance without listening. We found similar, but listening is still required.

% faster playback
% - Can use time compression (section 2.6), which works even better with transcript
% - Whittaker2002 included it
% - Vemuri2004 shows that adding transcripts increases maxiumum playback rate

% confidence shading
% - Suhm2001 found that confidence shading slowed down correction
% - Burke2006 reported that users find confidence shading helpful
% - Vemuri2004 found no statistically significant difference in comprehension with/without confidence shading

% transcript accuracy
%- Transcripts are sufficiently accurate for professional radio producton
% - ASR is sufficiently accurate for professional production
% - Yoon2014 found that transcripts were not accurate enough to use without listening
% - Sivaraman2016 found semantic editing ASR errors weren't a heavy distraction

%One of the findings from Chapter~\ref{chp:ethno} was that the observed radio producers used paper to allow for
%handwritten annotation and to facilitate face-to-face collaboration.
%Chapter~\ref{chp:paper}:
%- Screen and digital pen interfaces both work well for semantic speech editing, strengths in different situations
%- Paper is easier to read and simpler, making it better for simple edits with accurate transcripts
%- Screen has integrated playback and correction, which is better for complex editing and less accurate transcripts
%- Screen interface has remote collaboration, but pen might be better for face-to-face
%- Pen has greater flexiblity for portablity, but depends on access to printer
%- Lack of re-ordering and labelling in both mean it's only useful for rough editing

%- Transcripts not needed for short recordings, or where content is led by sound
% - short recordings can be kept in memory (15-25 mins)
% - actuality led programmes, emphasis on sound

% stage of usefulness
% - mostly useful for rough edit
% - continuing to further stages requires good integration with tools
% - EDL allows non-destructive integration 


%- Participants reported that using paper made it easier to read transcripts, remember information, think widely and
%  orient themselves. This aligns with previous research.

% - Editing with pen better for making simple edits and quick decisions, using familiar content with high-quality transcript
% - Editing with screen better for more complicated editing involving complex decisions using less familiar content
%   with a lower accuracy transcript
% - Collaboration - pen better for face-to-face, screen allows remote collab
% - Portability - pen is more flexible, allowing for working in more places away from screen, but relies on printer
% - Labelling and re-ordering features would be useful
% - Listened to process information, judge quality and identify non-speech sounds

% - Transcript accuracy affected need for correction, reading speed, reliance on listening, longevity and edit
%   granularity
% - Sivaraman2016 found that users only made fine-grain edits rather than rough edits



% LEARNINGS
%   - What do I know now, that I didn’t know before doing the research?
%   - What do I know that no one else knows? (e.g., things that arise from my unique context or data sets)


% COMPARE TO EXISTING WORK
%   - new research, positioned against existing knowledge
%   - Does your work confirm or contradict other published work?
%   - key aspects of the literature you have studied and say how they are justified or contradicted by your research


% LIMITATIONS
%   - stress the limitations of the current work


% CONTRIBUTIONS
%   - Give an overview of the main original contributions
%   - emphasise the contribution that it makes to research

  %A set of theoretical models of audio editing workflows in professional radio production, which
  %contribute to the academic understanding of radio production.

  %A demonstration of the feasibility of semantic audio visualization, and insights into the
  %effectiveness of audio waveforms, for segmenting music from speech.

  %The application of semantic speech editing to professional radio production, with
  %demonstration of its feasibility and insights into its limitations. 

  %A novel approach to editing speech recordings through the combination of semantic speech
  %editing and a digital pen interface. Insights into the relative benefits of paper and screen interfaces for semantic
  %speech editing.

  %Other outputs include: contributions towards the academic understanding of radio production workflows
  %and open-source implementations of tools for semantic speech editing and audio visualiation (see
  %Appendix~\ref{app:tech}).

% APPLICATIONS
%   - Who cares? / Who should care?
%   - Practical applications/implications
%   - a paragraph in which you reflect upon the practical implications of your results into the your field
%   - connect the dots in an attempt to lead the reader to a deeper understanding of the implications of the work

  % benefits and drawbacks of automation
  % - logging and cross-referencing could be assisted or automated
  % - this may remove the opportunity to listen back, which help thinking process
  % - serendipidy may play an important role in identifying high quality content

  % commercial interest
  % - Investment in Descript shows that there is market interest and value in semantic speech editing
  % - Avid have included ScriptSync since 2007

  % free up time/money
  %Radio producers have tight deadlines and reducing budgets. Tools which allow them to be more efficient can therefore
  %free up time and creative energy to improve the output of the programme.

  % enabling new types of programmes
  %Alternative methods of interacting with audio content may also give rise to opportunities for more innovative programme
  %making.
  %For example, the comedy duo known as ``Cassetteboy'' \citep{Perraudin2014} remix speeches to make popular figures say
  %controverial things, which could be achieved very easily using a semantic speech editor.
  % could be misused?

  % democratisation
  %Semantic editing tools may lower the barrier of entry to audio production.
  %Podcasting had democratized distribution of audio content, but digitial audio workstations are feature-rich
  %and have complicated layouts, which may be intimidating to novice users.
  %Semantic speech editors could make it easier for amateur users to create their own content, and the increasing
  %popularity in podcasts gives amateur producers an audience and method of distribution.
  % - Sivaraman2016 found semantic editing was more accessible than waveform editing
  % - Yoon2014 found that semantic editing was easy to use

  % correction could allow publication of transcripts
  % - correcton is less necessary when producer can remember what happened
  % - accuracy affects longevity of recordings as memory fades
  % - accuracy may affect granularity
  %NPR Shortcut\citet{Friedhoff2016}


% FURTHER WORK
%   - What questions does your research raise, and is there potential for further research?
%   - Recommendations for further research


%=====================================================================================================================


%It would likely be easier to create effective specific visualizations as they
%only need to include information relevant to the application, but this would make it difficult to cater for every
%possible use case. Users would also have to switch between multiple specific visualizations depending on the task they
%are performing, and learn how to read each of them. General visualizations would have to contain information relevant
%to all applications. It would be difficult to select the most relevant information and map it in a way that humans can
%easily interpret. However, they could be used for any purpose without users having to switch between or learn to read
%multiple visualizations.






\section{Further work}\label{sec:conclusions-further}

Further work that could follow on from the research of this thesis includes:

\paragraph{Collaborative semantic speech interaction:}
% use operational transformation to allow simultaneous access and editing
% NPR presedential debate fact-checking \citep{Fisher2016}

The semantic speech editing systems we developed during this thesis were designed to be operated by individuals.
However, as shown in Chapter~\ref{chp:ethno}, radio production involves teams of people.  Collaborative tools may help
these teams work together more efficiently.  Operational transformation techniques \citep{Sun2004} have enabled the
development of collaborative document editing tools, such as the popular Google Docs, which can facilitate concurrent
team-based working.  For example, \citet{Fisher2016} describes how NPR used Google Docs to enable over a dozen
producers to collaboratively fact-check the US Presidential Election debates live, using real-time ASR transcripts. By
linking the text to audio, a similar approach could be used for collaborative semantic editing of speech. Teams could
also use annotation to make suggestions for edits that are then accepted or rejected by the programme producer.
%TODO Integrated writing and editing, like Shin2016

\paragraph{Assisted/automatic de-umming:}
% take into account intonation

In Section~\ref{sec:doc}, we saw that a large proportion of a studio manager's time was spent on cleaning-up interview
material by removing unwanted vocal noises, known as ``de-umming''. Examples of unwanted noises include ``umm''s and
``err''s, long or loud breaths, and redundant phrases such as ``you know''.  These can be difficult to identify, as
their presence is not clearly visible using current tools, and they can be difficult to remove as they often overlap
and blend with the speech. \citet{Loviscach2013} demonstrated a prototype ``umm detector'' that extracted MFCC
\citep{Imai1983} features from the audio and used template matching to visually highlight potential ``umm''s. We could
not find any other work that attempted to detect or remove unwanted speech noises.

One solution could be to train an
ASR system to ``transcribe'' unwanted sounds. These ``words'' could then be highlighted, or removed in the same way
that transcripts can be semantically edited.  However, this assumes that these noises have clearly defined boundaries,
which they often do not.  In our experience, de-umming involves a level of editorial and creative judgement, so a
interface that supports a human-in-the-loop system may be necessary.
% - Rubin2013 and Berthouzoz2012 included de-umm features

\paragraph{Improved automatic speech recognition:}
% use prior information to produce better transcripts

In this thesis, we found that the transcripts produced by current ASR systems were sufficiently accurate to perform
semantic speech editing in professional radio production. However, as discussed in
Section~\ref{sec:transcript-generation}, ASR quality affects comprehension, search, and the need for correction, so it
is desirable to improve the transcript accuracy. Prior to transcription, much is already known about the content of the
speech, and by providing this information to the ASR system in advance, it could be used to improve the quality of the
transcript.  For example, the topic of conversation could be used to weight the likelihood of words related to that
topic appearing.  Many ASR systems have a speaker diarization pre-processing stage, and the number, gender and identity
of speakers could also be used to inform this process.

\paragraph{Improved digital pen system:}
% re-ordering and labelling
% integrated playback using wireless link
% correction
% better intepretation

The design of our digital pen semantic editor in Chapter~\ref{chp:paper} was influenced by the technical limitations of
the technology that we used to implement it.  This prevented us from including certain features such as integrated
listening and correction, and was inflexible in interpreting the user's gestures.  These issues could be solved by
using a different system for implementation.

The digital pen we used operated in batch mode as it connected to the
computer using a USB dock. This prevented us from integrating control over the playback of the audio. A digital pen
with a wireless link could allow a user to play, pause and navigate the audio using the paper transcript.  The design
of our paper layout used strict rectangular boundaries to capture edit commands, which created a potential source of
errors. By using a more flexible approach, the system may be able to better understand the user's intentions and avoid
inadvertent mistakes.  Additionally, the ability to distinguish between edit commands and handwriting could allow the
user to both edit and correct the transcript on paper.

\paragraph{Penless paper editor:}
% use scanner or camera, mask out printed area
% digital ink with electronic paper

In Chapter~\ref{chp:paper}, we based the design of our system on a digital pen as it combined the readability of real
paper with the digital capture of freehand annotations. However, when we evaluated the system, some producers were
concerned about losing the pen, having to share it and its cost. Several producers said they would prefer to use a
highlighter or different coloured ink, which digital pens do not support. There are also patents that cover digital
pens \citep{Fahraeus2003}, which can make development more difficult.

Scanners and cameras could be used as alternative methods of digitally capturing freehand annotations from paper.  For
example, a barcode or QR code could be used to label each page with a unique identity that links to a stored image of
the printed page. The stored image could be used to mask the picture/scan of the paper, leaving only the user's
annotations. Image analysis could then be used to interpret the user's annotations and translate them into edits,
corrections and labels.
%TODO E-paper interface

\paragraph{Rich audio visualization:}
% multiple features
% different colour spaces
% texture

In Chapter~\ref{chp:colourised}, we showed that speech/music segmentation can be performed faster, more easily and more
accurately by using pseudocolour to map a semantic audio scalar feature to a colour gradient. Previous systems from
\citet{Tzanetakis2000} and \citet{Mason2007} have also showed how false colour can be used to map multiple features to
colour space.  Onomatopoeia is the formation of words that resemble the sound they describe.  The author believes that
there is significant untapped potential to design an onomatopoeic audio visualization that ``looks like it sounds''.
Such a visualization could provide an efficient and accessible way of navigating all types of audio content.

Crossmodal correspondences could be exploited when designing the mapping between semantic audio features and visual
properties.  Many of the links listed in Section~\ref{sec:crossmodality} have yet to be tested for visually navigating
audio content, and previous work on audio visualization has mostly focused on colour.  Other visual properties such as
shape and texture could be used to increase the richness of the information presented in the image.  For example,
textures can be generated using procedural techniques \citep{Ebert1994}, which allows them to be synthesised from
semantic audio features.  However, selecting the best combination of semantic audio features and visual mappings is a
huge challenge, and is likely to be different for each task and signal.

%\paragraph{Cost/benefit of transcripts:}
%We found that semantic editing was faster for long recordings, but it would be
%interesting to investigate how long a recording needs to be before the benefits of semantic editing outweigh the
%costs.  Additionally, it is unclear how much of the speed benefit comes from automatically generating the transcript,
%and how much comes from the editing interface.

%\paragraph{Non-speech sounds}
%Tagging music and environmental sounds

\paragraph{Application to television}

The focus of research in this thesis has been solely on semantic audio tools for the production of radio. However,
there may be opportunities to transfer some of the tools and findings from this work to the production of television.
The weekly reach and consumption hours for television (91\%, 25 hours) is similar to that of radio (90\%, 21 hours)
\citep[pp.  82, 119]{Ofcom2017}.  However, based on the author's experience in working with BBC producers on both sides
of the fence, there are big differences between production in television and radio in terms of team size, budget and
culture. These differences may impact on the relevance of the work.

Radio is often produced by small teams of between two and five. Television production involves dozens or sometimes
hundreds of people, as demonstrated by the list of credits at the end of each programme.  In 2016/17, the BBC spent
\textsterling2.48B on television production -- over five times the amount that was spent on radio production \citep[pp.
39, 111]{Ofcom2017}. There are also important cultural differences between television and radio production. For
example, in the BBC radio producers are employed directly by the public service arm of the BBC as full-time staff. In
television, most producers are freelancers working on short-term contracts, either for an independent company or a
commercial arm of the BBC. It is likely that these differences will affect the effect and use of semantic audio
production tools, but this is currently unknown, so may be an interesting direction for the research.


% SPEAKER DIARIZATION 
%All of productions we observed may benefit from being able to see where different people are speaking in a recording,
%known as ``speaker diarization''. In drama, it could help producers identify different lines in the script, in
%documentaries it could highlight the position of the questions in interview recordings, and in news it could help
%producers navigate long off-air recordings. There is an existing body of work on speaker diarization
%\citep{AngueraMiro2012} that has been used to help users navigate radio archives \citep{Raimond2014}, but could also be
%used in the production process.

% PRE-RECORDED CONTENT
%Drama production involves recording multiple takes of the same content. This technique allows producers to get the most
%from the actors, but means that it can be difficult to select which performances to use.  Comparing takes during
%post-production is a manual and inefficient process.  Providing an easy way to directly compare performances could
%allow the director to make a better informed decision on which takes to use. Automatically aligning the takes on
%different tracks in the DAW would be an easy way to make such comparisons.


\clearpage
\section{Summary}\label{sec:conclusions-conclusions}

%New methods for the analysis of audio production workflows.
%A set of novel theoretical models of audio editing workflows that contribute to the academic understanding of
%professional radio production.

%The first formal study on the effect of audio waveforms and semantic audio visualization on user performance.
%Insights into the effectiveness of audio waveforms, for segmenting music from speech.

%The first application of semantic speech editing to professional radio production. Development of a semantic speech
%editor that includes a novel DAW integration. The first formal user study of semantic speech editing for audio
%production. Insights into the challenges and limitations of semantic speech editing in the context of radio
%production. 

%A novel approach to editing speech recordings through the combination of semantic speech editing and a digital pen
%interface. Insights into the relative benefits of paper and screen interfaces for semantic speech editing.

We started our research by conducted three ethnographic case studies of professional radio production. We found that
the radio producers used audio waveforms and textual representations to navigate and edit audio content, and used paper
to make freehand annotations and facilitate face-to-face collaboration.  Based on these findings, we developed and
evaluated three semantic audio tools for radio production using audio visualization, and semantic speech editing on 
screen and paper interfaces.

We developed a simple semantic audio visualization for segmenting music from speech and conducted the first formal
study on the effect of audio waveforms and semantic audio visualization on user performance. We found that our semantic
audio visualization was faster, more accurate and required less effort than using a waveform.  Using an audio waveform
required less effort than no visualization, but was not significantly faster, nor more accurate.

We developed a semantic speech editing system for radio production, and evaluated this approach with professional radio
producers for the production of genuine programmes. The radio producers were successful in using semantic speech
editing with ASR transcripts as part of their workflow, and continued to use our system after the study. Through our
study, we also gained insights into the importance of annotation, collaboration, portability and listening.

Finally, we designed and developed a novel system for semantic speech editing on paper using a digital pen. We compared
our paper interface to a screen interface and normal paper through a user study of professional radio producers. We
found that the benefits of reading from paper and the simplicity of the pen interface made it better for fast, simple
editing with familiar audio and accurate transcripts.  The integrated listening and correction features of the screen
interface made it better for more complex editing with less familiar audio and less accurate transcripts.  We also
gained insights into effect of ASR accuracy, the role of listening, and the relative benefits of paper and screen
interfaces for collaboration and portability.

%In this thesis, we have demonstrated three successful approaches to improving radio production.

