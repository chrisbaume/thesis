% !TeX root = main.tex
\chapter{Introduction}\label{chp:intro}

Radio broadcasting...

%The British Broadcasting Company, as it was known then, started broadcasting its first daily radio service in 1922.
90\% of the UK adult population listen to the radio each week for an average of 21 hours \citep{RAJAR2017a}. The
British Broadcasting Corporation (BBC) has the largest share of radio listening in UK. Each week, 64\% of the
population listen for an average of 16 hours to its 9 national and 40 local radio stations \citep{RAJAR2017a}.
Additionally, the BBC World Service broadcasts radio services in 29 languages to over 269M listeners per week worldwide
\citep{BBC2017}.
%There are 50 national radio stations in the UK \citep[p. 127]{Ofcom2017}

The format of radio programmes can be broadly categorised into music and speech.  In the UK, most radio listeners to
choose music programmes, but 38\% choose to listen to speech-based radio \citep[p.  97]{Ofcom2017}.

Podcasting is another form of audio listening, where pre-produced content is downloaded using the Internet.
Approximately 10\% of the UK adult population listen to podcasts \citep{RAJAR2017}. The BBC is the most popular
source of podcasts in the UK \citep[p. 107]{Ofcom2017}, with most podcasts .

%Community radio:
%Each community radio station relies on 72 volunteers on average.
%Staffing accounts for over half of all costs in community radio. \citep[p. 116]{Ofcom2017}

\section{Motivation}\label{sec:intro-motivation}

%\subsection{Sound}
% Sound is linear - has a beginning and an end, constrained by the arrow of time
% Cannot be searched or skimmed
% Sound is one-dimensional
% We never stop listening
% Huge range of frequencies and pressure levels

%Audio is an electronic representation of sound which stores pressure waves as an electrical signal.

% Characteristics of sound
One of the distinguishing characteristics of radio is that it is based exclusively on sound.  Although listeners have
no visual reference, sound stimulates the imagination and creates pictures in the mind's eye.  Radio is not limited
by the size of the screen in the way that television is. Sound design and music can be used to create scenes for
virtually any scenario, which may otherwise be impossible or too expensive to put on screen. As the old adage goes
``the pictures are so much better on the radio''.

Humans use sound to communicate through language and music, which can richly convey complex ideas and elicite powerful
emotion. Despite this, sound is simply the result of vibration in a medium such as air.  This requires movement, which
means that sound can only exist over a period of time. The temporal nature of sound gives it unique properties that
make it both an interesting and challenging medium to work with.

Unlike pictures, which can be viewed and searched at a glance, sound recordings must be perceived through listening.
The time needed to naturally listen to a sound recording is the same as the length of the recording. This means that
reviewing long recordings can take a large amount of time. Sound is also a linear medium that must be played in
sequence, which can make it challenging to navigate sound recordings non-linearly. 
%For example, playing sounds in reverse often makes them incomprehensible.

%From Dhanesha2010:
%It is a common practice for us to skim textual content on a web page. While skimming, we usually skip words or phrases
%that are not of interest to us and we slow down our speed when the content seems to be of relevance to us. But when we
%listen to audio content, which is not persistent and is sequential, such skimming is not possible.

Radio production is a process of recording, selecting and re-arranging audio content. The ability to efficiently
navigate the audio is crucial to the success of this process.  Modern radio production is performed using software
known as a ``digital audio workstation'', or DAW.

%- Modern digital audio workstations are general purpose and not well-suited to most radio production tasks (pic of
%cluttered DAW interface)

DAWs visualize audio by plotting the amplitude of the audio signal over time, known as an \textit{audio waveform}.
Waveforms allow users to interact with the audio spatially rather than temporally, which is thought to be a faster and
easier way to navigate audio recordings.  Waveforms display some useful information, but are limited in the information
they can convey.  For example, when viewing a waveform at the right scale, it is possible for an experienced user to
distinguish between speech and music, but it is not usually possible to determine the style of the music, or what is
being said.

%Audio visualization attempts to take advantage of the properties of the human visual system by mapping sound to images.
%This is done to allow listeners to see the sound in a way they can understand and make use of. This approach takes
%advantage of the spatial layout that is possible with images, and the natural searching and skimming capabilities of
%the human visual system.

%- Producers spend much of their time on menial tasks, which is wasteful because the time they spend on this is time
%they can't spend on making the programme better.

%To overcome the challenges that come with the temporal nature of sound, there are a number of technological solutions
%that can be applied. At the very simplest level, sound can be navigated using rewind and fast-forward, to allow
%listeners to re-listen or skip ahead to a sound recording. The playback rate of sound recordings can also be increased,
%but there are limitations to this, which we discuss in Section X. However, in order to navigate sound with purpose, it
%is preferable to be able to understand what is contained within the sound, and to be able to comprehend this
%information efficiently. There have been two main approaches to achieving this -- audio visualization and semantic
%audio analysis.

Semantic audio analysis is the task of deriving meaning from audio. This is achieved by extracting low-level audio
features, such as energy or frequency, and mapping them to a human-readable representation, such as categories or
words. Applying semantic audio analysis techniques to radio production tasks may allow us to produce richer user
interfaces that assist the production process.

%Semantic audio analysis refers to a collection of techniques for analysing sound signals to extract human-readable
%information from them. This can be used to map sound to high-level information, such as the name of the person
%speaking, or the text of what they are saying. These techniques can vary from low-level features such as the most
%dominant frequency, to high-level features such as the text of the words that are being spoken. These techniques can
%also be combined with visualization methods to display the results.

%Semantic audio analysis can extract useful information, but how can this best be used?

This research was partly inspired by a conference presentation from \citet{Loviscach2013}, who demonstrated several
prototypes that used semantic audio analysis to assist the editing of recorded speech. These included visualization of
vowels using colour, detecting and highlighting ``umm''s, and identifying repetition. These prototypes were developed
as new features for navigating and editing lecture recordings using custom video editing software
\citep{Loviscach2011}.

Developing richer methods of interacting with audio content may allow radio producers to read more about the content of
the audio may make it easier and faster for them to produce their radio programme.
We were interested to discover whether this approach could be used to improve the radio
production process, and which techniques worked better than others.

As part of this work, we want
to understand how these techniques can best be applied to the production of radio, and make the process more efficient
by reducing the time that is needed to produce the programme. 

%Sound naturally occurs at a consistent rate, so everybody experiences sound over the same length of time.
%This makes it a great medium for shared
%experiences, the kind of which can be witnessed at music festivals. However,

%Working with sound alone makes radio production a much simpler affair than with other media. For example, an interview
%can be recorded using a portable sound recorder by an individual without much training. This simplicity makes radio
%production accessible to amateurs, and the introduction of podcasting has also made it easier for amateurs to
%distribute their own content.

%- transitory nature of radio means it can only be heard at the time of broadcast, no second chances
%Historically, the temporal nature of sound and radio mean that listeners could only hear a programme at the time it was
%broadcast, and once it was heard it was gone. Nowadays, the Internet has facilitated the introduction of on-demand
%listening and podcasting that have changed the wa

%Sound is defined as vibrations that travel through a medium, typically air, which humans perceive through their ears.
%As it is based on vibration, sound cannot exist in a moment -- it can only take place over a period of time.
%Sound is linear, as it is based on a sequence of motion.
%It is experienced at a consistent rate, which cannot naturally be increased or desreased.

% challenges of sound
%This time-based nature of sound gives it a fascinating character.
%However, the time-based nature of sound comes with a variety of challenges.
%Sound cannot be experienced instantanously, as it is the change in air pressure which produces sound.
%As such, it takes time to consume sound.
%For consuming long sound recordings, this property means that listening requires an investment of time.
%In radio production, much of the time required to produce a programme must be committed to listening to the sound that has been recorded.
%The purpose of listening to the sound includes a variety of reasons, including reminding themselves of what
%was said, working out whether the sound quality is acceptable, deciding which bits of the recording to use, and many
%more.

%- Although radio production has turned digital, it has not undergone a technological revolution and the workflow
%remains almost exactly the same

%- Digital communication has allows producers greater flexibility and quality in getting content, but the editing
%process is still analogous to cutting and pasting tape


% BBC situation
The research presented in this thesis was conducted during, and as part of, the author's employment at BBC Research and
Development.  BBC R\&D exists to promote technological innovation that supports the BBC's mission to enrich people's
lives with programmes and services that inform, educate and entertain \citep[art. 15]{BBCCharter2016}.  This is
achieved through the research and development of broadcast technology, including for the production and distribution of
audio content.

Part of the motivation behind this research is to make radio production a more efficient process, without affecting the
output. This would free up resources that could either be spent on producing higher quality content, or used to making
financial savings. The BBC spent \textsterling471M on radio production in 2016/17 \citep[p. 111]{Ofcom2017}, so even
minor improvements could make large savings.  We would also like to make radio production a more enjoyable and creative
experience, where producers spend less time on boring, menial tasks and more time on activities that contribute to the
quality of the programme output.

The author's position within the BBC gives us privileged access to production staff and working environments that would
otherwise be inaccessible to researchers. We see this as a rare opportunity to conduct research that directly involves
professional radio producers.

% live vs pre-produced
Most radio is broadcast live, where the sound is mixed and produced as the audience listeners to it.  Although there
are hours of planning and preparation that go into a live radio broadcast, the actual audio production happens in
real-time, so there is no opportunity to make the audio production more efficient.  However, many radio programmes are
pre-produced, where the sound is recorded before transmission.  In these cases, more audio is recorded than is
required, and audio editing is used to reduce the length, remove unwanted sounds and re-arrange the material into the
final programme. This process creates overheads for the producers.

\section{Aim and scope}

The aim of this work is to develop and evaluate methods for radio production that improve the production process, such
as by increasing the efficiency, reducing the effort required or opening up new creative possibilities.

% - do not include tools for automatically editing (computer based decisions)
%Radio production is a creative endeavour and 
The intention behind this research is to facilitate creative expression, rather than replace it through automation.
The aim is to find ways for machines and humans to work to each of their strengths, where simple and menial tasks are
automated, but there is always a ``human in the loop'' that decides on the final output.  Our hope is that in addition
to making production activities more efficient, this may unlock opportunities for greater creative expression.

% make use of BBC producers
We are targeting this work specifically as radio production, and as such, the methods we develop should be evaluated on
radio producers. We intend to make the most of the access to professional radio producers that is available to us by
involving them in the development and evalutation where possible.

% - speech, not music
Music-based radio programmes are more popular, but we will focus our work on speech radio as most original radio
content is speech-based.

% - not including live production, which is already efficient
This editing process creates overheads for the producers, so there are opportunities to make this audio production
process more efficient.  For this reason, we chose to focus our research on radio programmes that are fully or partly
produced in advance of broadcast, and in particular on the audio editing stage of radio production.

\section{Thesis structure}\label{sec:intro/structure}

\paragraph{Chapter \ref{chp:background}} introduces the previous research which we will build our work upon. We start
by giving a general overview of radio production and semantic audio to provide context to our research. We then survey
related techniques and previous systems that have attempted to assist the navigation and editing of audio. These are
categorized into audio visualization, textual representation and audio playback interfaces. We conclude by reflecting
on this existing work to consider a strategy for achieving our research aim.

\paragraph{Chapter \ref{chp:ethno}} investigates audio editing workflows in radio production at the BBC.  This study
aims to better understand the roles, environment, tools, tasks and challenges involved in real-life radio production,
to help inform the direction of our research.  We achieve this by conducting three ethnographic case studies
of news, drama and documentary production.  The results of this study highlight three opportunities for further
research, which we explore individually in the following chapters.

\paragraph{Chapter \ref{chp:colourised}} investigates the effect that audio visualization has on radio production. We
conducted an online task-based quantitative study in which participants segmented music from speech under three
conditions.

presents a quantitative study that looked at how the waveform performs in a
common production tasks, and whether it can be improved.

\paragraph{Chapter \ref{chp:screen}} uses a qualitative study to investigate how a screen-based semantic audio editing
interface affects the production process.

\paragraph{Chapter \ref{chp:paper}} describes a qualitative study that investigates how a paper-based semantic audio
editing interface compares to the screen-based approach and normal paper.

\paragraph{Chapter \ref{chp:conclusions}} concludes the thesis and considers the prospects for future research.

\section{Contributions}\label{sec:intro-contributions}

The principal contributions of this thesis are:
\begin{itemize}
  \item Chapter~\ref{chp:ethno}: 
  \item Chapter~\ref{chp:colourised}: 
  \item Chapter~\ref{chp:screen}: 
  \item Chapter~\ref{chp:paper}:
    A novel approach to editing speech recordings by using a digital pen and a printed transcript.
\end{itemize}

\section{Associated publications}\label{sec:intro-publications}

Portions of the work detailed in this thesis have been presented in the following publications:

\subsection*{Conference papers}

\begin{itemize}
  \item Chapter~\ref{chp:ethno}: Published and presented at the 138th Audio Engineering Society convention in Warsaw,
    Poland \citep{Baume2015}
\end{itemize}

\subsection*{Software releases}
As part of this research, we have developed and released the following systems as open-source software:

\begin{itemize}
  \item \textbf{Dialogger}: A semantic speech editing interface (see Appendix~\ref{sec:dialogger})
  \item \textbf{Vampeyer}: A plugin framework for generating audio visualizations from the output of Vamp plugins
    (see Appendix~\ref{sec:vampeyer})
  \item \textbf{BeatMap}: A Javascript user interface element for navigating audio using audio visualizations
    (see Appendix~\ref{sec:beatmap})
\end{itemize}

